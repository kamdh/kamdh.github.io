\let\latexnofiles\nofiles
\let\nofiles\relax
\documentclass[margin,line]{res}
\usepackage[colorlinks=true, linkcolor=blue,citecolor=blue, urlcolor=blue, breaklinks=true]{hyperref}
\usepackage{breakurl}
\usepackage{graphicx}% Include figure files
\usepackage{fullpage}
\usepackage{etaremune}


\pagestyle{empty}

\pdfpagewidth 8.5in
\pdfpageheight 11in 

\setlength\oddsidemargin{-.6in}
\setlength\evensidemargin{-.6in}
\setlength\textwidth{6.1in}
%\setlength\sectionwidth=in
\setlength\itemsep{0in}
\setlength\parsep{0in}
\setlength\parskip{0.1in}
\setlength\textheight{10in}
\setlength\topmargin{-.5in}
\setlength\footskip{1in}

\newenvironment{list1}{
  \begin{list}{$\cdot$}{%
      \setlength{\itemsep}{0in}
      \setlength{\parsep}{0in} \setlength{\parskip}{0in}
      \setlength{\topsep}{0in} \setlength{\partopsep}{0in} 
      \setlength{\leftmargin}{0.17in}}}{\end{list}}
\newenvironment{list2}{
  \begin{list}{$\bullet$}{%
      \setlength{\itemsep}{0in}
      \setlength{\parsep}{0in} \setlength{\parskip}{0in}
      \setlength{\topsep}{0in} \setlength{\partopsep}{0in} 
      \setlength{\leftmargin}{0.2in}}}{\end{list}}

\begin{document}
\name{Kameron Decker Harris \hspace{2.5cm} Curriculum Vitae \vspace*{.1in}
\hspace{2.5cm}
 {\small \normalfont Updated: \today} }

\begin{resume}

  \section{\sc Contact}
  Email: \href{mailto:kameron.harris@wwu.edu}{kameron.harris@wwu.edu}\quad
  Web: \href{http://glomerul.us}{glomerul.us}\quad
  Code:
  \href{https://github.com/kamdh/}{kamdh},
  \href{https://github.com/glomerulus-lab}{glomerulus-lab}
  % \begin{tabular}{@{}p{3.5in}p{.4in}p{1.25in}}
  %   % Paul G.\ Allen School of Computer Science \& Engineering\\
  %   % University of Washington
  %   Box 352350, 185 E Stevens Way NE
  %   % Cell: +56 (09) 82 42 84 56 
  %   & Email: & \href{mailto:kameron.harris@wwu.edu}{kameron.harris@wwu.edu}\\ 
  %   Seattle, WA
  %   98195-2350 
  %   & Web:   & \href{http://glomerul.us}{glomerul.us} \\
  %   & Code:  & \href{https://github.com/kamdh/}{github.com/kamdh}
  %   % \vspace{-2cm} \includegraphics[scale=0.4]{danforth.jpg} \\ 
  % \end{tabular}

  \section{\sc Positions}

  {\bf Assistant Professor},
  2020--present\\
  Department of Computer Science,
  Western Washington University, Bellingham, Washington.

  %\section{\sc Former}
  {\bf Washington Research Foundation Postdoctoral Fellow},  
  2018--2020\\
  Paul G.\ Allen School of Computer Science \& Engineering and
  Department of Biology\\
  University of Washington, Seattle, Washington.



  \section{\sc Research Interests}

  I am a computational scientist and applied mathematician 
  with an interest in {\em \bf neural systems} and
  {\em \bf networks/graph theory}. 
  These systems are often adaptive, meaning they can
  learn from their environment 
  and exhibit complex behavior.
  I enjoy collaborative, interdisciplinary research at the
  intersection of theory, modeling, and data science,
  including work in machine learning and analytics.

  % {\bf Neuroscience:} 
  % Structure-function relationships in neural circuits. 
  % I am developing novel machine learning methods to infer 
  % neural connectivity and segment behaviors 
  % from sparsely sampled experimental data.
  % Implementing these is leading to challenging optimization
  % and linear algebra problems in high dimensions.
  % Other areas of interest: synchrony among 
  % inhibitory and excitatory neurons and its control;
  % the importance of network constraints on biological 
  % and artificial learning systems.

  % {\bf Networks:}
  % Processes on networks such as contagion, percolation, and oscillations.
  % Random graph models built with realistic statistics such as
  % degrees, correlations, and community structure.
  % Connections between computer science, graph theory, and brain science.
  % Social networks and data science.

  % {\bf Computational social science:}
  % Social networks such as Twitter provide an unprecedented amount of
  % data about individuals' behavior online.
  % These data allow us to quantify large-scale trends by studying the
  % content of online posts and the variable social network
  % in which they are embedded.
  % An open question is the amount that classical reduced models of
  % social dynamics actually explain real human activity.

  % \section{\sc Current Research}

  % % Computational neuroscience: 
  % % \begin{list1}
  % %   \item  Network inference from Allen Institute for Brain Science
  % %     tracing data; 
  % %     with Nicholas Cain, Stefan Mihalas and Eric Shea-Brown.
  % %   \item Models of respiratory rhythm generation; with 
  % %     Tatiana Dashevskiy, Joshua Mendoza, 
  % %     Jan-Marino Ramirez, and Eric Shea-Brown.
  % % \end{list1}

  % % Random graphs: 
  % % \begin{list1}
  % %   \item Tree approximations, spreading, and spectral theory;
  % %     with Gerandy Montes de Oca and Ioana Dumitriu.
  % % \end{list1}
  % {\bf Computational neuroscience:}\\
  % Network inference from Allen Institute for Brain Science
  % tracing data; 
  % with Stefan Mihalas and Eric Shea-Brown.
  
  % Models of respiratory rhythm generation; with 
  % Tatiana Dashevskiy, Joshua Mendoza, 
  % Jan-Marino Ramirez, and Eric Shea-Brown.

  % {\bf Random graphs:} \\
  % Tree approximations, spreading, and spectral theory;
  % with Gerandy Brito and
  % Ioana Dumitriu.
      
  \section{\sc Publications}

  A list of publications including citations and metrics is on \href{https://scholar.google.com/citations?user=ClJ1ExwAAAAJ&sortby=pubdate}{Google Scholar}.
  
  {\setlength{\leftmargini}{0pt}
    \begin{etaremune}
    %\begin{list1}
      % \item KD Harris, PS Dodds. 
      %   \textit{Branching processes in the theory of spreading on complex networks}.
      %   In preparation. 2014.
      % \item KD Harris, PS Dodds, JL Payne.
      %   %\href{http://arxiv.org/abs/1108.5398}
      %   {\textit{Direct, physically motivated derivation of
      %   triggering probabilities for spreading
      %   processes on generalized random networks}}.
      %   In preparation. 2014.
    \item M Xie, S Muscinelli, KD Harris, A Litwin-Kumar.
      \textit{\href{https://www.biorxiv.org/content/10.1101/2022.08.15.504040}{Task-dependent optimal representations for cerebellar learning}}.
      In review, eLife.
    \item S Daetwiler, A Read, J Stillwell, KD Harris.
      \textit{\href{https://arxiv.org/abs/2205.02291}{BrainViewer: interacting with spatial connectome data at the mesoscale}}.
      Preprint.
    \item B Pandey, M Pachitariu, BW Brunton, KD Harris.
      \textit{\href{https://www.biorxiv.org/content/10.1101/2021.09.09.459651}{Structured random receptive fields enable informative sensory encodings}}.
       PLoS Comput Biol 18(10): e1010484. 2022.
    \item KD Harris, Y Zhu.
      \textit{\href{https://arxiv.org/abs/1910.10692}{Deterministic tensor completion with hypergraph expanders}}.
      SIAM Mathematics of Data Science 3(4), 1117--1140. 2021.
    \item KD Harris, A Aravkin, R Rao, BW Brunton.
      \textit{\href{https://arxiv.org/abs/1905.08389}{Time-varying Autoregression with Low Rank Tensors}}.
      SIAM Applied Dynamical Systems 20(4), 2335--2358. 2021.
    \item G Brito, I Dumitriu, KD Harris.
      \textit{\href{https://doi.org/10.1017/S0963548321000249}{
        Spectral gap in random bipartite biregular graphs 
        and applications}}.
    Combinatorics, Probability, \& Computing, 1--39. 2021.
    \item WS DeWitt, KD Harris, AP Ragsdale, K Harris.
      \textit{\href{https://www.biorxiv.org/content/10.1101/2020.06.16.153452v1}{Nonparametric coalescent inference of mutation spectrum history and demography}}.
      Proceedings of the National Academy of Sciences
      118 (21) e2013798118. 2021.
    \item SM Hirsh, KD Harris, JN Kutz, BW Brunton.
      \textit{\href{https://doi.org/10.1137/19M1289881}{Centering Data Improves the Dynamic Mode Decomposition}}.
      SIAM Applied Dynamical Systems 19(3), 1920--1955. 2020.
    \item KD Harris. 
      \textit{\href{https://arxiv.org/abs/1909.02603}{Additive function approximation in the brain}}. 
      Workshop paper: Real Neurons and Hidden Units.
      Neural Information Processing Systems. 2019.
    \item P K{\"u}rschner, S Dolgov, KD Harris, P Benner.
      \textit{\href{https://arxiv.org/abs/1808.05510}{
          Greedy low-rank algorithm for smooth connectome regression}}.
      Journal of Mathematical Neuroscience 9: 9. 2019.
    \item J Knox, KD Harris, N Graddis, JD Whitesell, H Zeng, JA Harris,
      E Shea-Brown, S Mihalas.
      \textit{\href{https://doi.org/10.1162/netn_a_00066}{
          High resolution data-driven model of the mouse connectome}}.
      Network Neuroscience 3(1), 217-236. 2019.
    \item KD Harris, T Dashevskiy, J Mendoza, AJ Garcia III, 
      J-M Ramirez, E Shea-Brown.
      \textit{\href{https://arxiv.org/abs/1610.04258}{
          Different roles for inhibition in the rhythm-generating respiratory network}}.
      Journal of Neurophysiology 118(4), 2070--2088. 2017.
    \item A Litwin-Kumar, KD Harris, R Axel, H Sompolinsky, LF Abbott.
      \textit{\href{http://dx.doi.org/10.1016/j.neuron.2017.01.030}{
          Optimal degrees of synaptic connectivity}}.
      Neuron 93, 1153--1164. 2017.
    \item KD Harris, S Mihalas, E Shea-Brown. 
      \textit{\href{https://arxiv.org/abs/1605.08031}{
          High resolution neural connectivity from 
          incomplete tracing data using nonnegative spline regression}}.
      Neural Information Processing Systems, 2016.
    \item PS Dodds, EM Clark, S Desu, MR Frank, AJ Reagan, JR Williams, 
      L Mitchell, KD Harris, IM Kloumann, JP Bagrow, K Megerdoomian, 
      MT McMahon, BF Tivnan, CM Danforth.
      \textit{\href{https://arxiv.org/abs/1406.3855}{Human language 
          reveals a universal positivity bias}}.
      Proceedings of the National Academy of Sciences 112(8), 2389--2394. 
      2015.
    \item KD Harris, PS Dodds, CM Danforth.
      \textit{\href{https://arxiv.org/abs/1303.1414}
      {Dynamical influence processes on networks:
          General theory and applications to social contagion}}.
      Physical Review E 88, 022816. 2013.
    \item L Mitchell, MR Frank, KD Harris, PS Dodds, CM Danforth.
      \textit{\href{https://arxiv.org/abs/1302.3299}
      {The Geography of Happiness:
          Connecting Twitter sentiment and expression,
          demographics, and objective characteristics of place}}.
      PLoS ONE 8(5): e64417. 2013.
    \item PS Dodds, KD Harris, CM Danforth.
      \href{https://arxiv.org/abs/1208.0255}
      {\textit{Limited Imitation Contagion on Random Networks: 
          Chaos, Universality, and Unpredictability}}.
      Physical Review Letters 110, 158701. 2013.
    \item CA Bliss, IM Kloumann, KD Harris, CM Danforth, PS Dodds.
      \href{https://arxiv.org/abs/1112.1010}
      {\textit{Twitter reciprocal reply networks exhibit
          assortativity with respect to happiness}}.
      Journal of Computational Science 3(5), 388--397. 2012.
    \item KD Harris, E-H Ridouane, DL Hitt, CM Danforth.
      \href{https://arxiv.org/abs/1108.5685}
      {\textit{ Predicting flow reversals in chaotic natural
          convection using data assimilation}}.
      Tellus A 64, 17598. 2012.
    \item N Allgaier, KD Harris, CM Danforth.
      \href{https://arxiv.org/abs/1107.2690}
      {\textit{Empirical Correction of a Toy Climate Model}}. \\
      Physical Review E 85, 026201. 2012.
    \item IM Kloumann, CM Danforth, KD Harris, CA Bliss, PS Dodds.
      \href{https://arxiv.org/abs/1108.5192}
      {\textit{Positivity of the English language}}.
      PLoS ONE 7(1): e29484. 2012.
    \item PS Dodds, KD Harris, IM Kloumann, CA Bliss, CM Danforth.
      \href{https://arxiv.org/abs/1101.5120}
      {\textit{Temporal patterns of happiness and information in a 
          global social network:
          Hedonometrics and Twitter}}. 
      PLoS ONE 6(12): e26752. 2011.
    \item JL Payne, KD Harris, PS Dodds.
      \href{https://arxiv.org/abs/1103.0056}
      {\textit{Exact solutions for social and biological contagion models 
          on mixed directed and undirected, degree-correlated random networks}}.
      Physical Review E 84, 016110. 2011.
    \item PS Dodds, KD Harris, JL Payne.
      \href{https://arxiv.org/abs/1101.5591}
      {\textit{Direct, physically motivated derivation of the contagion
          condition for spreading processes on generalized random networks}}.
      Physical Review E 83, 056122. 2011.
      % \item KD Harris, M Munizaga, A Gschwender.
      %   \textit{Nocturnal bus timetabling in the citywide 
      %   transit network Transantiago}.
      %   In preparation. 2011.
    \end{etaremune}
    %\end{list1}
  }

  % \section{\sc Articles in Preparation}
  % {
  %   \setlength{\leftmargini}{0pt}
  % }

  \section{\sc Grants and Honors} 
  
  Washington Research Foundation Postdoctoral Fellowship, 
  3 yr.\ grant,
  2018--2020

  {Graduate}
  \begin{list1}
  \item
    Boeing Research Award in Applied Mathematics,
    University of Washington, 2016
  \item 
    Big Data for Genomics and Neuroscience Training Grant,
    University of Washington,
    %\$22,920/year and 60\% tuition, 
    % 2 yrs.\ funding,
    2015--2017
  \item
    Joseph Hammack Research Award in Applied Mathematics, 
    University of Washington, 2014
  \item 
    Boeing Fellowship, University of Washington, 
    %\$30k/year stipend,
    3 yrs.\ funding,
    2012--2015
  \item 
    John F.\ Kenney Award for Excellence in Mathematics, 
    University of Vermont, 2012
  \item
    NASA Graduate Research Assistantship, University of Vermont,
    %\$25k 1-year stipend, 
    2009, 2011
  \end{list1}

  {Fulbright Scholar}, Chile, 2010
  \begin{list1}
    \item Worked with transportation engineers to optimize the nighttime
      bus timetables of Santiago
  \end{list1}

  {Undergraduate}
  \begin{list1}
  \item
    Vermont Scholar, 
    funded 100\% financial need,
    %covered tuition and stipend at 100\% of financial need, 
    2005--2009
  \item
    URECA! NASA-funded undergraduate research grant, 2008--2009
  \item
    Honors College Scholar, \textit{Magna cum laude}, Phi Beta Kappa
  \item 
    Senior Mathematics Award for Research, 2009
    % \item
    %   GPA 3.93 cumulative, 4.00 in mathematics, and 3.95 in physics; 
    % \item
    %   GRE general scores: 690 verbal, 800 quantitative
    % \item
    %   GRE subject score: 660 mathematics
  \end{list1}

  % {High School}
  % \begin{list1}
  % \item
  %   Certificate of merit in the statewide UVM Math Contest, 2003 and 2004
  % \end{list1}
  \section{\sc Education}

  {\bf University of Washington}, Seattle, Washington.
  \begin{list1}
    \item Ph.D., Applied Mathematics, December 2017. \\
      Thesis: 
      \textit{\href{http://hdl.handle.net/1773/40831}
        {This Brain Is a Mess: Inference, Random Graphs, and
          Biophysics to Disentangle Neuronal Networks}}.
      Advisor: Eric Shea-Brown\\
      Committee: Ioana Dumitriu, Adrienne Fairhall, Stefan Mihalas, 
      Jan-Marino Ramirez
    %\item M.S., Applied Mathematics, June 2013.%\\
      %GPA: 3.76/4.00
  \end{list1}
  {\bf University of Vermont}, Burlington, Vermont.
  \begin{list1}
  \item M.S., Mathematics, October 2012.\\
    Thesis: 
    \textit{\href{https://arxiv.org/abs/1209.2177}
      {On-off Threshold Models of Social Contagion}}.\\
    Advisor: Peter Sheridan Dodds.%\\
    %GPA: 3.95/4.00
  \item B.A., Mathematics and B.A., Physics, May 2009.\\
    Honors College thesis: 
    \textit{
      \href{https://faculty.washington.edu/kamdh/thermosyphon/thesis.pdf}
      {Predicting Climate Regime Change in Chaotic Convection}}.\\
    Advisor: Christopher M. Danforth.%\\
    %GPA: 3.93/4.00
  \end{list1}
  % {\bf Santa Fe Institute}, Santa Fe, New Mexico.\\
  % Complex Systems Summer School, 2011.

  % {\bf Pontif\'{i}cia Universidad Cat\'{o}lica de Valpara\'{i}so}, 
  % Valpara\'{i}so,	Chile\\
  % ISEP Exchange Program, July 2007 to January 2008.

  Other education
  \begin{list1}
  \item Methods in Computational Neuroscience. Marine Biological Lab, Woods Hole, MA. 2015.
  \item  Mining and modeling of neuroscience data. 
    Redwood Center for Theoretical Neuroscience. \\
    University of California Berkeley, 2013.
  \item Complex Systems Summer School. Santa Fe Institute, 2011.
  \item ISEP Exchange Program. 
    Pontif\'{i}cia Universidad Cat\'{o}lica de Valpara\'{i}so, Chile. 2007.
  \end{list1}

  \section{\sc Teaching \&\\Mentoring}

  Instructor
  \begin{list1}
  \item \href{https://www.cneuro.net/cneuro2022}{CNeuro2022}.
    Theoretical and Computational Neuroscience Summer School.\\
    IOB (Basel, Switzerland) and Tsinghua University (Beijing, China)
    \hfill
    August 2022
  \item Machine Learning I (DATA 371) \hfill Spring 2022
  \item Analysis of Algorithms II (CSCI 405) \hfill Fall 2021
  \item Analysis of Algorithms I (CSCI 305)
    \hfill Winter, Spring 2021; Spring, Fall 2022
  \item \href{https://glomerul.us/teaching/CSCI-471/2020-Fall/}{Machine Learning}
    (CSCI 471/571) \hfill Fall 2020
  \item Data Science for Biologists (UW BIO 419/519) \hfill Winter 2019
  \item \href{https://glomerul.us/teaching/422-522/}
    {Introduction to Computational Models in Biology} (UW AMATH 422/522) 
    \hfill Fall 2017
  \end{list1}

  Research mentoring
  \begin{list1}
  \item Vivian White, CS undergrad. 2022--now.
  \item Suyhun ``Michael'' Ban, CS undergrad. 2021--now.
  \item Angus Read, CS grad. 2021--now.
  \item Caitlin Bannister, Neuroscience undergrad. 2021--now.
  \item Jessica Stillwell (now: PNNL), CS undergrad. 2020--2022.
  \item Grant Chou
    (now: \href{https://faculty.washington.edu/tuthill/people.html}{Tuthill lab} research tech),
    CS undergraduate. 2020--2022.
  \item Biraj Pandey, Applied Math PhD student. 2019--2022.
  \item Sean McCulloch (now: Allen Institute for Brain Science), 
    WWU CS MSc student. 2020--2021.
  \item Seth Hirsh (now: Facebook Reality labs),
    Physics PhD student. 2018--2020.
  \item Satpreet Singh, Electrical \& Computer Engineering PhD student. 2018--2019.
  \item Yuchen Wang (now: Adobe), CS \& Engineering undergrad. 2018--2019.
  \item Nathan Lee, Applied Math PhD student. 2018--2019.
  \item Joseph Knox (now: Facebook data scientist),
    Allen Institute for Brain Science. 2017--2018.
  \item Nile Graddis, Allen Institute for Brain Science. 2015--2018.
  \item Joshua Mendoza (now: PNNL data scientist).
    Applied Math Honors thesis:
    {\it The effects of network structure in creating a two-phase 
      breathing pattern in the B\"otzinger and pre-B\"otzinger complexes},
    2014--2015.
    UWIN Postbac Fellowship, 2016.
  \end{list1}

  Teaching assistant
  \begin{list1}
  \item \href{https://www.youtube.com/watch?v=OmYkj1FImpI}{Summer Workshop on the Dynamic Brain}
    (UW/Allen Institute for Brain Science)
    \hfill Summer 2016
  \item Women in Science and Engineering undergraduate program, UW
    \hfill Summer 2014
  \item STEM Bridge undergraduate program, UW
    \hfill Summer 2013
  \item Introduction to Nonlinear Dynamics and Chaos (UW AMATH 402/502)
    \hfill Winter 2013, 2014
  \item Fundamentals of Calculus I (UVM MATH 19) \hfill Fall 2008
  \item Basic Combinatorial Theory (UVM MATH 173) \hfill Spring 2007
  \end{list1}

  Instructor, Champlain Valley Union High School ACCESS community classes
  \begin{list1}
  \item Intermediate Spanish: Advanced
    \hfill Spring 2012
  \item Spanish for Beginners
    \hfill Fall 2011
  \end{list1}

  Guest lecturer
  \begin{list1}
  \item Computational Modeling of Biological Systems (UW AMATH 422/522)
    \hfill Fall 2016
  \item Introduction to Nonlinear Dynamics and Chaos (UW AMATH 402/502)
    \hfill Winter 2015
  \item Linear Algebra (UVM MATH 124) \hfill Fall 2009
  \item Introduction to Numerical Analysis (UVM MATH 237) \hfill  Fall 2009
  \item Chaos and Fractals (UVM MATH 266) \hfill Spring 2009
  \end{list1}

  Tutor
  \begin{list1}
  \item UVM Learning Coop ``in-house'' undergraduate tutor 
    \hfill December 2006--March 2009
  \item VSAC Tutor Academy training \hfill 2008
  \end{list1}

  Ski instructor, Mad River Glen Freeride, students age 11--18 \hfill
  Winter 2009--2010

  \section{\sc Service and\\Professional Societies}

  CS graduate program committee, 2021--now

  Member: Society for Industrial and Applied Mathematics,
  Organization for Computational Neuroscience

  Conference organizer: \href{https://deepmath-conference.com/}{DeepMath} 2022
  
  Conference referee:
  COSYNE (2021, 2022),
  NeurIPS (2021, 2022),
  SIAM Workshop on Network Science (2018)
  
  Referee:
  Network Neuroscience,
  NeuroImage,
  Physical Review Letters, PLOS ONE, PLOS Comp Bio,
  Physical Review E, New Media and Society

  Volunteer, SIAM Math Fair at Lockwood Elementary School, 2014

  Computational neuroscience journal club co-organizer, 2014--2016

  Organized UW Applied Math ski day, 2015

  \section{\sc Selected Talks}

  KD Harris.
  \textit{Network geometry for sensing and learning}.
  \begin{list1}
  \item \href{https://www.youtube.com/watch?v=le5itcrXmCo}{CAIDA AI seminar}, University of British Columbia invited talk
  \item NeuroAI in Seattle 2022, invited talk
  \end{list1}

  KD Harris.
  \textit{A Random Perspective on Tuning Properties of Sensory Neurons}.
  Invited talk, Center for Integrative Brain Research,
  Seattle Children's Research Institute, Seattle, WA. 2021.
  
  KD Harris.
  \textit{Rigorous theories of neural nets and kernel methods}.
  Invited talk, Computational Neuroscience Seminar,
  University of Washington, Seattle, WA. 2021.
  
  KD Harris.
  \textit{\href{https://glomerul.us/presentations/2021-05-12_leveraging_lacuna.pdf}{Leveraging the lacuna: spectral gaps and tensor recovery}}.
  Invited talk, Minisymposium: 
  Latest advances in spectral linear algebra in network science.
  SIAM Linear Algebra. 2021.

  KD Harris.
  \textit{\href{https://glomerul.us/presentations/2020-06-17_kernel_neuroscience-expanded_notes.pdf}{Kernel theories of networks and their use in neuroscience}}.
  Computational Neuroscience Seminar, University of Washington, Seattle, WA. 2020.

  KD Harris, Y Zhu.
  \textit{\href{https://glomerul.us/presentations/2020-05-05_tensor_complexity_completion-expanded.pdf}{Tensor complexity and completion using hypergraph expanders}}.
  Theory seminar. 
  Computer Science \& Engineering, University of Washington,
  Seattle, WA. 2020.

  KD Harris.
  \textit{Brain network reconstruction as an inverse problem}.
  Invited talk, Workshop: Inverse Problems in Imaging.
  SIAM PNW 2019, Seattle, WA. 2019.

  KD Harris, I Dumitriu, G Brito.
  \textit{\href{https://glomerul.us/presentations/2019-01-15_random_graph_theory_seminar-expanded.pdf}
    {Spectral gap in random bipartite biregular graphs and applications}}.
  Theory seminar.
  Computer Science \& Engineering, University of Washington,
  Seattle, WA. 2019.

  KD Harris, J Harris, S Mihalas, H Choi, J Whitesell. 
  \textit{The Mouse Mesoscale Connectome: Data-driven Models and Organization of
Cortical Networks}.
  Showcase Symposium team talk.
  Allen Institute for Brain Science, Seattle, WA. 2018.

  KD Harris, A Litwin-Kumar, R Axel, H Sompolinsky, LF Abbott.
  \textit{\href{https://glomerul.us/presentations/2018-07-18_optimal_connectivity.pdf}
    {Connections between dimensionality and network sparsity}}.
  Invited talk, Workshop: 
  How does learning reshape dimensionality of collective network activity?
  Computational Neuroscience (CNS) 2018, Seattle, WA. 2018.

  KD Harris.
  \textit{Advances and challenges in connectome inference and analysis}.
  \begin{list1}
  \item Institute for Disease Modeling, Bellevue, WA. 2017.
  \item Pacific Northwest National Laboratory, Richland, WA. 2017.
  \end{list1}
 
  KD Harris, N Graddis, J Harris, S Mihalas, ET Shea-Brown.
  \textit{\href{https://glomerul.us/presentations/2016-04-01_AIBS_Synergy.pdf}
    {Construction of a voxel-based mesoscopic mouse connectome}}.
  Synergy Lecture. 
  Allen Institute for Brain Science, Seattle, WA. 2016.

  KD Harris, T Dashevskiy, J Mendoza, AJ Garcia III, J-M Ramirez, and ET Shea-Brown.
  \textit{\href{https://glomerul.us/videos/201512101124-Banff-Harris.mp4}
    {Role and limits of inhibition in an excitatory burst generator}}.
  Invited talk, 
  Connecting Network Architecture and Network Computation.
  BIRS, Banff, AB. 2015.

  KD Harris, T Dashevskiy, J-M Ramirez, and ET Shea-Brown.
  \textit{Combined Effects of Connectivity and Inhibition in a Model of 
    Breathing Rhythmogenesis}. 
  SIAM Dynamical Systems Conference. Snowbird Resort, 
  Little Cottonwood Canyon, UT. 2015.

  KD Harris, PS Dodds, CM Danforth.
  \textit{Limited Imitation Social Contagion as a Model of Fashions}.
  \begin{list1}
  \item Dynamics Days 2013. Marriott City Center, Denver, CO. 2013.
  \item SIAM Dynamical Systems Conference. Snowbird Resort,
    Little Cottonwood Canyon, UT. 2013.
  \end{list1}

  KD Harris, M Munizaga, A Gschwender.
  \textit{Timetable optimization for Transantiago night service---Fulbright
    studies in Chile}.
  Transportation Research Center Brown Bag Series.
  University of Vermont, Burlington, VT. 2011.

  KD Harris, PS Dodds, CM Danforth, IM Kloumann, CA Bliss.
  \textit{\href{https://glomerul.us/presentations/2011-04-25_twitter_geo_happiness.pdf}
    {Geographical variation of happiness as expressed by users of Twitter}}.
  Student Research Conference.
  University of Vermont, Burlington, VT. 2011.

  KD Harris, M Munizaga, A Gschwender.
  \textit{\href{https://glomerul.us/presentations/2010-12-15_fulbright.pdf}
    {Programaci\'{o}n de servicios nocturnos---Proponiendo
      un horario optimizado para Transantiago}}
  (Programming nightly bus service---Proposing 
  an optimized schedule for Transantiago).
  Seminar ``Herramientas avanzadas para la ciudad del futuro.''
  Universidad de Chile, Santiago, Chile. 2010.
  
  KD Harris, E-H Ridouane, D Hitt, CM Danforth. \\
  \textit{\href{https://glomerul.us/presentations/2011-05-21_siam_thermosyphon.pdf}
    {Forecasting Flow Reversals in a Chaotic Toy Climate}}. 
  \begin{list1}
  \item SIAM Dynamical Systems Conference. Snowbird Resort,
    Little Cottonwood Canyon, UT. 2011.
  \item Applied Mathematics Seminar. University of Vermont, 
    Burlington, VT. 2009.
  \item Graduate Student Seminar. 
    University of Vermont, Burlington, VT. 2009.
  \item Student Research Conference.
    University of Vermont, Burlington, VT. 2009.
  \item MAA Northeastern Sectional Meeting. St. Michael's College, 
    Burlington, VT. 2008.
  \item URECA! Awards Ceremony. 
    University of Vermont, Burlington, VT. 2008.
  \end{list1}

  % \section{\sc Co-authored \\ Talks}

  % PS Dodds, CM Danforth, IM Kloumann, KD Harris, CA Bliss.
  % \textit{Patterns of Happiness in Social Network of Twitter}.
  % Northwestern University, Text as Data Conference, Evanston, IL. 2011.

  % CM Danforth, R Lieb-Lappen, NA Allgaier, KD Harris.
  % \textit{Dynamical Systems Approaches to Climate Prediction}.
  % AMS Meeting, Worcester, MA. 2011.

  % CM Danforth, KD Harris, NA Allgaier, E-H Ridouane. \\
  % \textit{Forecasting Chaotic Physical Processes}.
  % \begin{list1}
  % \item Dartmouth College, Mathematics Colloquium, Hanover, NH. 2009.
  % \item Cornell University, Institute for Computational Sustainability,
  %   Math \& Climate. 2009.
  % \item UC Berkeley Climate Change Summer School, MSRI, 
  %   Berkeley, CA. 2008.
  % \item Bates College, Department of Mathematics, Lewiston, ME. 2008.
  % \item National Weather Service, Burlington, VT. 2009.
  % \end{list1}

  % CM Danforth, KD Harris, NA Allgaier, E-H Ridouane. 
  % \textit{Chaos and the Mathematics of Prediction: Harry Potter, 
  %   Hurricane Katrina, and Happiness}. 
  % MAA Northeastern Sectional Dinner Meeting, Simmons College, Boston, MA. 2008.


  \section{\sc Technical Skills} 
  Programming languages: Python (scipy, numpy, networkx, scikit-learn, pandas),
  MATLAB, C/C++

  Data: Experience with PostgreSQL + PostGIS, Hadoop + Pig, R

  Computer systems: cluster environments, Linux, \LaTeX, git

  Spoken language: Spanish (fluent), English (native), French (basic)

  % \section{\sc Personal Interests}
  % Skiing 
  % (various backcountry ski descents in Cascade Range 
  % including North face of Northwest ridge, Mt.\ Adams, 
  % Wy'East face, Mt.\ Hood,
  % Park headwall, Mt.\ Baker, 
  % and Emmons glacier, Mt.\ Rainier;
  % former Freeride coach at Mad River Glen;
  % captained high school ski team), 
  % frisbee, hiking, biking, climbing, 
  % mountaineering 
  % (Diente del Diablo, Parva del Inca, Volc\'{a}n San Jos\'{e}, and
  % Volc\'{a}n Licancabur, Chile; 
  % North face Mt.\ Maude, Mts.\ Triumph and Forbidden, Washington)

  \section{\sc References}
  {\bf Bingni Wen Brunton}\\
  Associate Professor,
  Department of Biology,
  University of Washington\\
  \href{mailto:bbrunton@uw.edu}{bbrunton@uw.edu}
  
  {\bf Eric Shea-Brown}\\
  Professor,
  Department of Applied Mathematics,
  University of Washington\\
  \href{mailto:etsb@amath.washington.edu}{etsb@amath.washington.edu}
  
  {\bf Ioana Dumitriu}\\
  Professor,
  Department of Mathematics,
  University of California San Diego\\
  \href{mailto:dumitriu@ucsd.edu}{dumitriu@ucsd.edu}

  {\bf Christopher M. Danforth}\\
  Professor, 
  Department of Mathematics \& Statistics,
  University of Vermont\\
  \href{mailto:cdanfort@uvm.edu}{cdanfort@uvm.edu}

  {\bf Peter Dodds}\\
  Professor,
  Department of Computer Science,
  University of Vermont\\
  \href{mailto:pdodds@uvm.edu}{pdodds@uvm.edu}

  % {\bf Stefan Mihalas}\\
  % Assistant Investigator,
  % Allen Institute for Brain Science\\
  % \href{mailto:stefanm@alleninstitute.org}{stefanm@alleninstitute.org}


  

  % {\bf Adrienne Fairhall}\\
  % Professor, 
  % Department of Physiology and Biophysics,
  % University of Washington\\
  % \href{mailto:fairhall@u.washington.edu}{fairhall@u.washington.edu}  


  

\end{resume}
\end{document}

